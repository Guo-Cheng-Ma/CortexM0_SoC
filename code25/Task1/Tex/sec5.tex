\section{实验总结}

本次实验通过“汇编计算、C 调用汇编、C 内嵌汇编”三个任务,理解了 Cortex-M0 启动流程、寄存器与栈的用法以及 ATPCS 调用约定,学会在 Keil 中单步调试和查看寄存器、内存变化,并在多次解决链接错误和库配置问题的过程中体会到环境配置与问题定位的重要性。例如在实验中先后遇到 \verb|__initial_sp| 和 \verb|__user_initial_stackheap| 未定义、启动文件重复、编译器 V5/V6 选择不当、是否启用 MicroLIB 等问题,最终通过查阅文档和分析报错信息逐一解决。总体来说,本次实验既加深了我对 Cortex-M0 架构和调用约定的理解,也让我实际体验了一遍“环境搭建—出错—定位—修复”的完整调试流程,为后续进一步学习嵌入式系统开发打下了基础。

