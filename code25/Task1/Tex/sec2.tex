\maketitle
\section{实验原理} % [cite: 3]
\subsection{Cortex-M0 调试环境}
本实验使用 Keil MDK 集成开发环境进行软件开发与调试。Cortex-M0 处理器内置 CoreSight 调试技术,支持通过仿真器(如 CMSIS-DAP 或 J-Link)进行硬件调试。调试环境的核心功能包括:
\begin{itemize}
    \item \textbf{寄存器观察}:实时监控通用寄存器(R0-R12)、栈指针(SP)、链接寄存器(LR)和程序计数器(PC)的状态变化。
    \item \textbf{内存映射(Memory Map)}:利用 ARM 的统一编址特性,通过 Memory 窗口直接查看代码段、数据段及栈空间的具体数值。
    \item \textbf{执行控制}:支持设置硬件断点、单步运行(Step)和全速运行,便于追踪程序流程。
\end{itemize}
\newpage
\subsection{指令工作过程}
Cortex-M0 处理器基于 ARMv6-M 架构,主要执行 16 位的 Thumb 指令。指令执行遵循 \textbf{三级流水线} 机制:
\begin{enumerate}
    \item \textbf{取指 (Fetch)}:从存储器中读取指令代码。
    \item \textbf{译码 (Decode)}:解析指令含义及操作数。
    \item \textbf{执行 (Execute)}:执行 ALU 运算、数据传输或跳转操作。
\end{enumerate}
\begin{figure}[htbp]
    \centering
    \begin{tikzpicture}[
        node distance=2.8cm,
        >=Stealth,
        every node/.style={font=\small},
        box/.style={
            draw,
            rounded corners,
            minimum width=2.8cm,
            minimum height=1.2cm,
            align=center
        }
    ]
        % 三个阶段
        \node[box] (fetch)  {取指\\(Fetch)};
        \node[box, right=of fetch] (decode) {译码\\(Decode)};
        \node[box, right=of decode] (exec)  {执行\\(Execute)};

        % 箭头
        \draw[->] (fetch) -- (decode);
        \draw[->] (decode) -- (exec);

        % PC 的说明(可选)
        \node[above=0.6cm of fetch] (pc) {PC 指向当前指令地址 + 4};
        \draw[->] (pc) -- (fetch);
    \end{tikzpicture}
    \caption{Cortex-M0 指令三级流水线示意图}
\end{figure}

由于流水线的存在,PC (R15) 的值通常等于当前执行指令地址加 4(两条指令的偏移)。

\subsection{函数调用的基本原理}
ARM 汇编程序与 C 语言程序的相互调用遵循 \textbf{ATPCS (ARM-Thumb Procedure Call Standard)} 标准:
\begin{itemize}
    \item \textbf{参数传递}:函数的具体参数依次通过寄存器 \textbf{R0-R3} 传递;若参数超过 4 个,剩余参数通过栈传递。
    \item \textbf{返回值}:函数的返回值通常保存在 \textbf{R0} 寄存器中。
    \item \textbf{栈 (Stack)}:Cortex-M0 使用\textbf{满递减 (Full Descending)}栈。栈指针 SP (R13) 指向当前栈顶数据,入栈 (PUSH) 时地址减小,出栈 (POP) 时地址增加。栈用于保存局部变量及函数调用时的现场(Context)。
    \item \textbf{返回地址}:使用 \texttt{BL} 指令调用函数时,下一条指令的地址会自动保存到链接寄存器 \textbf{LR (R14)} 中。函数返回时,通常将 LR 的值加载回 PC,或通过 \texttt{POP \{PC\}} 实现返回。
    \item \textbf{寄存器保护}:被调用函数需保证 \textbf{R4-R11} 的值在调用前后保持不变(Callee-saved),若需使用这些寄存器,必须先压栈保护。
\end{itemize}