

\section{实验过程} % [cite: 14]
\textbf{提示:此处要求对程序的工作原理进行分析,以及解释为什么这么修改示例程序,并且通过Keil调试模式的截图证明结果的正确性。}

\subsection{任务一:汇编编程计算}

本任务要求在 Cortex-M0 平台上编写一个简单的汇编程序,实现给定算术表达式的计算,并通过 Keil 调试器观察寄存器中结果的变化,加深对 ARM 指令执行过程和调用约定的理解。

本实验工程包含\textbf{提供的启动文件} \verb|startup_CMSDK_CM0.s| 与\textbf{自编汇编源文件} \verb|compute.s|。

启动文件主要负责系统复位后的环境初始化与中断向量表定义。具体流程如下:首先在 \verb|STACK| 段中分配 1\,KB 的栈空间并导出栈顶符号 \verb|__initial_sp|;其次在 \verb|RESET| 段构建向量表 \verb|__Vectors|,将栈顶地址与复位中断服务程序 \verb|Reset_Handler| 填入表项前两项。

在代码段实现的 \verb|Reset_Handler| 中,程序通过 \verb|IMPORT main| 声明外部符号,并执行 \verb|LDR R0, =main| 与 \verb|BX R0| 指令,将控制权直接转移至用户编写的 \verb|main| 函数。值得注意的是,该启动文件直接跳转至用户入口 \verb|main| 而非标准 C 库入口 \verb|__main|,这意味着程序跳过了 C 运行库的初始化过程(如全局变量初始化);但为了满足链接器对符号的静态依赖,我在自编的 \verb|compute.s| 中额外定义了桩函数 \verb|__user_initial_stackheap|,从而保证了工程的顺利编译与链接。

\verb|compute.s| 中给出了任务一的具体计算逻辑。程序首先定义代码段并导出符号:
\begin{verbatim}
    AREA    MyTask1, CODE, READONLY
    EXPORT  main
    EXPORT  __user_initial_stackheap
\end{verbatim}

其中 \verb|main| 为程序入口,\verb|__user_initial_stackheap| 用于满足运行库对堆初始化符号的依赖(见下文)。在 \verb|main| 中,按照“先装载操作数,再执行乘加运算”的思路,依次完成如下操作:
\begin{enumerate}
  \item 使用 \verb|LDR| 指令将常数装载到通用寄存器:\verb|R0, = 2023310103|,\verb|R1, = 8|,\verb|R3, = 45|;
  \item 执行 \verb|MULS R0, R1, R0| 完成乘法运算,使 \verb|R0 = R1 * R0|,即 \verb|R0 = 2023310103 * 8|;
  \item 执行 \verb|ADDS R0, R0, R3| 完成加法运算,使 \verb|R0 = R0 + R3|,最终结果为
        \[
            R0 = 2023310103 \times 8 + 45 ;
        \]
  \item 使用 \verb|MOV R3, R0| 将结果再备份一份到 \verb|R3|,便于在调试时交叉检查;
  \item 最后跳转到标号 \verb|Stop|,并通过 \verb|B Stop| 形成死循环,使程序停在固定位置,便于在调试器中暂停和观察寄存器状态。
\end{enumerate}

为解决链接阶段运行库对 \verb|__user_initial_stackheap| 符号的依赖,我在文件末尾添加了一个最小实现:
\begin{verbatim}
__user_initial_stackheap
    BX      LR
\end{verbatim}

该函数本身不做任何实际堆初始化操作,但可以满足运行库对符号的需求,不会影响程序行为,因为本实验并不使用 \verb|malloc| 等需要堆空间的函数。

在 Keil 调试器中,我将程序运行至 \verb|Stop| 死循环处,并在寄存器窗口中观察结果。此时寄存器 \verb|R0| 的值为 \verb|0xC4CA18E5|,换算为十进制为
\[
    0xC4CA18E5 = 3\,301\,578\,981,
\]

而算术表达式的数学结果为
\[
    2023310103 \times 8 + 45 = 16\,186\,480\,869.
\]

通用寄存器宽度为 32 位,按无符号数解释时的范围是 $0\sim 2^{32}-1$,寄存器中实际保留的是上述结果对 $2^{32}$ 取模的值:
\[
    16\,186\,480\,869 \bmod 2^{32} = 3\,301\,578\,981,
\]

与寄存器中的数值完全一致;同时 \verb|R3| 中也保存了相同的结果。通过单步执行 \verb|MULS| 和 \verb|ADDS| 指令,可以直观地观察到 \verb|R0| 中数值的逐步变化。以上调试结果表明,任务一的汇编程序正确实现了题目要求的算术运算,程序设计和运行过程均符合预期。

\newpage
\begin{lstlisting}[language=Assembler, caption={任务一汇编程序代码}]
    ;compute.s
    AREA    MyTask1, CODE, READONLY             

    EXPORT  main            
    EXPORT  __user_initial_stackheap

main                            
    LDR     R0, =2023310103 
    LDR     R1, =8       
    LDR     R3, =45    

    MULS    R0, R1, R0      

    ADDS    R0, R0, R3      

    MOV     R3, R0          

Stop
    B       Stop            
    
    ALIGN

__user_initial_stackheap
    BX      LR              

    END
\end{lstlisting}
\begin{figure}[H]
    \centering
    % 一行四个
    \begin{subfigure}[b]{0.23\textwidth}
        \centering
        \includegraphics[width=\linewidth]{1.png}
        \caption{}
        \label{fig:2-a}
    \end{subfigure}\hspace{-0.3em}
    \begin{subfigure}[b]{0.23\textwidth}
        \centering
        \includegraphics[width=\linewidth]{2.png}
        
        \caption{}\label{fig:2-b}
    \end{subfigure}\hspace{-0.3em}
    \begin{subfigure}[b]{0.23\textwidth}
        \centering
        \includegraphics[width=\linewidth]{3.png}
        
        \caption{}\label{fig:2-c}
    \end{subfigure}\hspace{-0.3em}
    \begin{subfigure}[b]{0.23\textwidth}
        \centering
        \includegraphics[width=\linewidth]{4.png}
       
        \caption{} \label{fig:2-d}
    \end{subfigure}
    
    \caption{实验结果图示}
\end{figure}

如图 \ref{fig:2-a} 所示,$b, c, d$ 写入寄存器 $R_0, R_1, R_3$;图 \ref{fig:2-b} 所示,执行乘法指令后,$R_0$ 中存储 $a = b \times c$ 的结果;图 \ref{fig:2-c} 所示,执行加法指令后,$R_0$ 中存储 $a = b \times c + d$ 的结果;图 \ref{fig:2-d} 所示,寄存器 $R_0, R_3$ 中均存储 $a = b \times c + d$。

考虑到这里由于班级号太大导致溢出,可以考虑适当缩小之,比如写成3,这样计算结果就不会溢出了。这里不做展示。

\subsection{任务二:C程序调用汇编 (strcopy)}

\subsubsection{任务目标与原理分析}

本任务要求实现 C 语言调用汇编程序完成字符串复制(目标是将字符串修改为个人学号),并验证结果的正确性。

\begin{itemize}
    \item \textbf{程序工作原理:} 程序执行始于 $main()$ 函数。在 $main()$ 函数中,通过外部声明 \lstinline|extern void strcopy(char *d, const char *s);| 调用汇编函数 \lstinline|strcopy|。
    \item \textbf{传参机制 (ATPCS - ARM-Thumb Procedure Call Standard):} 根据 $ATPCS$ 规则,C 语言函数调用时,前四个 $32$ 位参数依次通过寄存器 $R0$ 至 $R3$ 传递。因此:
    \begin{itemize}
        \item 目的字符串地址(\lstinline|dststr|)作为第一个参数,存放在寄存器 $R0$ 中。
        \item 源字符串地址(\lstinline|srcstr|)作为第二个参数,存放在寄存器 $R1$ 中。
    \end{itemize}
    \item \textbf{汇编实现:} 汇编函数 \lstinline|strcopy| 利用寄存器间接寻址,循环地将 $R1$(源地址)指向的字符加载到 $R2$,然后将 $R2$ 存储到 $R0$(目的地址)。通过判断字符是否为 $0$(空字符 $\backslash 0$)来确定字符串的结束。
    \item \textbf{返回机制:} 子程序执行完复制操作后,使用 $BX\ lr$ (Branch and Exchange to Link Register) 指令返回到调用者($main$ 函数)。
\end{itemize}

\subsubsection{示例程序代码}

本任务主要使用了 \lstinline|main.c| 和 \lstinline|strcopy.s| 两个文件。

\begin{lstlisting}[language=Assembler, caption={汇编字符串复制代码}]
            ;strcopy.s 
            AREA SCopy, CODE, READONLY
            EXPORT  strcopy
strcopy
            LDRB     r2,     [r1]    ; R1 (源地址) 读取字符到 R2
            ADDS     r1,     #1      ; 源地址指针递增
            STRB     r2,     [r0]    ; R2 字符存储到 R0 (目的地址)
            ADDS     r0,     #1      ; 目的地址指针递增
            CMP      r2,     #0      ; 判断是否为 NULL 结束符
            BNE      strcopy         ; 非 NULL 则继续循环
            BX       lr              ; 汇编子程序返回
            END
\end{lstlisting}
\newpage
\begin{lstlisting}[language=C, caption={C语言调用汇编函数代码}]
//main.c
#include <string.h>
#include <stdint.h>
#include <stdio.h>

extern void strcopy(char *d, const char *s);

int main()
{
    char srcstr[] = "2023310103008";    
    char dststr[] = "Second string - destination";  
    strcopy(dststr,srcstr); 
    return 0;
}
\end{lstlisting}

\subsubsection{实验调试过程与结果验证}

在 Keil 调试器中,我们通过观察寄存器和内存窗口来验证程序的执行流程和结果。
\begin{figure}[H]
    \centering
    \includegraphics[width=0.8\linewidth]{5.png}
    \caption{初始状态}
    \label{fig:initial_state}
\end{figure}
如图 \ref{fig:initial_state} 所示,内存显示$Ascii$码“Second string - destination”存储在目的字符串地址处。此时$RO=0x20000430,R1=0x2000044C$

\begin{figure}[H]
    \centering
    \includegraphics[width=0.8\linewidth]{6.png}
    \caption{开始复制}
    \label{fig:start_copy}
\end{figure}

如图 \ref{fig:start_copy} 所示,开始复制后,寄存器 $R0=0x20000433,R1=0x20000450$,$R0$指向正在粘贴的位置,$R1$指向将要复制的位置

\begin{figure}[H]
    \centering
    \includegraphics[width=0.8\linewidth]{7.png}
    \caption{复制完成}
    \label{fig:copy_complete}
\end{figure}

如图 \ref{fig:copy_complete} 所示,复制完成后,内存显示目的字符串已被修改为“2023310103008”,验证了字符串复制的正确性。此时,由于检测到字符串结束符,标志位 $Z$ 被置为 1,导致条件跳转指令(\texttt{BNE})失效,从而跳出循环,标志着复制操作的结束。

\subsection{任务三:C语言内嵌汇编}

本任务要求在 C 语言中插入少量汇编代码,实现简单算术运算,并通过调试器观察寄存器与变量之间的对应关系,从而理解参数传递约定和内联汇编的用法。实验中使用的 C 程序如下所示:{\small
\begin{lstlisting}[language=C]
//main.c
#include <stdio.h>

int math_calc(int i, int j, int k)
{
    int res; 
    
    __asm
    {
        MOVS res, i      
        MULS res, j      
        ADDS res, res, k 
    }
    
    return res;
}

int main()
{
    int result;
    result = math_calc(2, 3, 4); 
    
    return 0;
}
\end{lstlisting}}

\paragraph{工作原理分析} 在 \verb|math_calc| 函数中,首先按照 C 语言的形式定义了三个整型参数 \verb|i|、\verb|j|、\verb|k| 和一个局部变量 \verb|res|。函数体内部使用 \verb|__asm { ... }| 块插入 ARM 汇编指令。编译器会在生成目标代码时,将 \verb|i|、\verb|j|、\verb|k|、\verb|res| 映射到对应的寄存器或栈上的临时单元,在内联汇编块中,标识符 \verb|i|、\verb|j|、\verb|k|、\verb|res| 就可以像普通寄存器一样被汇编指令访问。

内联汇编块完成的运算过程为:
\begin{enumerate}
  \item \verb|MOVS res, i|:将形参 \verb|i| 的值复制到局部变量 \verb|res| 中,作为后续计算的初始值;
  \item \verb|MULS res, j|:执行乘法运算 \verb|res = res * j|,此时 \verb|res = i * j|;
  \item \verb|ADDS res, res, k|:执行加法运算 \verb|res = res + k|,最终得到表达式
        \[
            res = i \times j + k .
        \]
\end{enumerate}
内联汇编结束后,\verb|res| 的值通过 \verb|return res;| 作为函数返回值返回。根据 ARM 的 ATPCS 约定,函数返回值最终会存放在寄存器 \verb|R0| 中,因此在调试器中观察到 \verb|R0| 的值即为计算结果。

在 \verb|main| 函数中调用 \verb|math_calc(2, 3, 4)|,期望结果为
\[
    2 \times 3 + 4 = 10 .
\]

\begin{figure}[H]
    \centering
    \includegraphics[width=0.3\linewidth]{8.png}
    \caption{初始赋值}
    \label{fig:initial_assignment}
\end{figure}

\begin{figure}[H]
    \centering
    \includegraphics[width=0.3\linewidth]{9.png}
    \caption{$i\times j$}
    \label{fig:乘法计算}
\end{figure}

\begin{figure}[H]
    \centering
    \includegraphics[width=0.3\linewidth]{10.png}
    \caption{得出结果}
    \label{fig:加法计算}
\end{figure}
在 Keil 调试模式下,单步执行到 \verb|return 0;| 前暂停,此时在寄存器窗口可以看到 \verb|R0 = 0x0000000A|,与十进制 10 一致;在 Watch 窗口中,\verb|result| 的值也为 10。调试结果验证了内联汇编块正确实现了给定的算术表达式,说明 C 语言与内嵌汇编之间的数据传递和寄存器使用均符合预期。


通过本实验,可以直观地理解:
\begin{itemize}
  \item C 形参和局部变量如何被映射到寄存器/栈空间,并在内联汇编中被访问;
  \item ARM 汇编中 \verb|MOVS|、\verb|MULS|、\verb|ADDS| 等指令的具体作用;
  \item 函数返回值通过寄存器 \verb|R0| 传递这一 ATPCS 规则在实际代码中的体现。
\end{itemize}
这为后续在更复杂的 C 工程中插入优化用汇编代码打下了基础。
