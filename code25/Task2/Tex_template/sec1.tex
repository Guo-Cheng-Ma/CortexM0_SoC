\section{实验目的}
\begin{enumerate}
    \item 搭建一个简单的SoC。掌握在总线上添加外设的方法,理解SoC系统的启动过程,总线读写操作的原理。
    \item 能编写简单的汇编程序,实现对CPU寄存器的控制,理解CPU运行的过程。
\end{enumerate}
\section{实验原理}
\begin{figure}[H]
    \includegraphics[width=0.8\textwidth]{1.png}
    \centering
    \caption{AHB-Lite 总线示意图}

\end{figure}


本实验搭建了一个基于 Cortex-M0 内核的简化 SoC。
Cortex-M0 作为唯一总线主机,通过 AHB-Lite 总线访问片上存储器 RAMCODE 以及预留的外设地址空间。
总线扩展模块 \texttt{bus\_ext} 内部包含地址译码器和从机多路选择器:地址译码器根据 \texttt{HADDR} 产生各从机片选信号 \texttt{HSELx},从机多路选择器根据片选结果选择对应从机的 \texttt{HRDATA}、\texttt{HRESP}、\texttt{HREADYOUT} 返回给 CPU。
RAMCODE 被映射到统一地址空间的低地址区域,在仿真开始时通过 \texttt{\$readmemh} 将 \texttt{code.hex} 预加载到 RAMCODE 中,作为 CPU 的指令和数据存储器。

实验采用 AHB-Lite 协议,一次传输分为地址阶段和数据阶段,并支持流水线。
在地址阶段,CPU 在时钟上升沿输出 \texttt{HADDR}、\texttt{HWRITE}、\texttt{HTRANS}、\texttt{HSIZE} 等信号,同时由地址译码器产生相应的 \texttt{HSELx}。
在数据阶段,写操作由 CPU 在 \texttt{HWDATA} 上提供写数据,读操作由从机在 \texttt{HRDATA} 上返回读数据,当 \texttt{HREADY} 为 1 时当前传输完成。
若从机未准备好,可通过将 \texttt{HREADYOUT} 拉低插入等待周期,此时 \texttt{HREADY} 为 0,总线保持当前传输状态,直到 \texttt{HREADY} 再次为 1。

仿真开始时,先将 \texttt{code.hex} 加载到 RAMCODE,对指令存储空间初始化。
释放复位后,Cortex-M0 按复位向量从 RAMCODE 中取出入口地址,设置 PC 并开始执行程序。
在运行过程中,CPU 通过 AHB-Lite 不断从 RAMCODE 取指,完成取指、译码和执行循环;执行 \texttt{LDR}/\texttt{STR} 等指令时,CPU 通过总线对 RAMCODE 或其它映射区域发起读写访问。
通过观察仿真波形中 \texttt{HADDR}、\texttt{HWRITE}、\texttt{HWDATA}、\texttt{HRDATA}、\texttt{HREADY} 等信号变化,可以清晰看出 CPU 复位启动、取指以及数据读写的时序过程。
