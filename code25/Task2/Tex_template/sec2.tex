\section{实验过程}
\subsection{$\texttt{vis\_r0\_o}$}
\begin{figure}[H]
    \centering
    \includegraphics[width=0.8\textwidth]{2.png}
    \caption{$\texttt{vis\_r0\_o}$实验结果}
\end{figure}

由图可见,\texttt{vis\_r0\_o}变化范围为0到4

\subsection{$\texttt{vis\_pc\_o}$}
\begin{figure}[H]
    \centering
    \includegraphics[width=0.8\textwidth]{3.png}
    \caption{$\texttt{vis\_pc\_o}$实验结果}
\end{figure}

由图可见,\texttt{vis\_r0\_o}变化范围为25到29。变化顺序如下:25-26--27-28-29-26-27-28-29-26-27-28-29-26-27-28。

在 “code.txt” 的 .text 中有这样一段:
\begin{lstlisting}
    start
        0x00000048:    2104        .!      MOVS     r1,#4
        0x0000004a:    2000        .       MOVS     r0,#0
        0x0000004c:    1c40        @.      ADDS     r0,r0,#1
        0x0000004e:    4288        .B      CMP      r0,r1
        0x00000050:    d0fb        ..      BEQ      0x4a ; start + 2
        0x00000052:    d1fb        ..      BNE      0x4c ; start + 4
\end{lstlisting}

这表明, 程序运行时, 取指地址在 $\texttt{0x0000004a-0x00000052}$ 之间循环, 不难发现, $\texttt{vis\_pc\_o}$ 恰好是取指地址去除 LSB 后的结果.

\subsection{AHB Lite 总线波形}
\begin{figure}[H]
    \centering
    \includegraphics[width=1\textwidth]{4.png}
    \caption{AHB Lite 总线波形}
    \label{fig:ahb_waveform}
\end{figure}

由图\ref{fig:ahb_waveform},$\texttt{HWRITE}$一直为低电平,说明一直是读操作;$\texttt{HBURST}$一直为低电平,说明一直是单次传输;尤其注意到光标所指的位置,可见$\texttt{HADDR}=0x4c$时长两个cycle,但是第一次上升沿结束后$\texttt{HRDATA}$仍然是不定态,说明此时从RAM读取数据还未准备好,直到第二个上升沿时$\texttt{HRDATA}$才变为有效数据$\texttt{0x1c40}$,这说明RAM有一个cycle的读延迟.这跟图\ref{fig:ram_read_waveform}所强调的两个阶段:Address phase $\&$ Data phase是一致的.

\begin{figure}[H]
    \centering
    \includegraphics[width=0.8\textwidth]{5.png}
    \caption{RAM 读基本波形}
    \label{fig:ram_read_waveform}
\end{figure}
\section{实验总结}
通过本次实验,掌握了AHB Lite总线的基本读写操作,理解了总线与RAM的交互过程,熟悉了汇编指令的执行流程,并通过仿真验证了设计的正确性,为后续更复杂的SoC设计打下了坚实的基础。

同时,最近我的Modelsim软件出问题了,在助教的建议下使用Vivado自带的仿真工具进行仿真,虽然界面和操作习惯上有些不同,但是也能仿照指导书进行添加波形观察信号的操作,学习到了新方法。之前为了观测已经例化模块里的信号,我都是在testbench里添加额外的输出端口,这次我学会了直接在仿真工具里添加信号进行观察,更加方便快捷。

\section{思考题}

\begin{enumerate}
  \item \textbf{怎样判断 CPU 处在 Reset 异常状态?}

  在 Cortex-M 处理器中,Reset 被视为一种异常,其异常号为 1,对应的异常入口为 \texttt{Reset\_Handler}。CPU 处在 Reset 异常状态时,实际上就是在执行 \texttt{Reset\_Handler} 对应的指令序列。

  \paragraph{(1) 从异常号寄存器角度} Cortex-M 内核提供了异常程序状态寄存器 IPSR(Interrupt Program Status Register)。当 CPU 正在执行某个异常服务程序(Exception Handler)时,IPSR 中会保存当前异常的异常号。若
  \[
      \texttt{IPSR} = 1,
  \]
  则表示当前处于 Reset 异常服务程序中,即 CPU 正在执行 \texttt{Reset\_Handler},此时可以认为 CPU 处在 Reset 异常状态。

  \paragraph{(2) 从 PC 值和向量表角度(结合本实验)}
  Cortex-M 的向量表(Vector Table)通常放在地址 \texttt{0x00000000} 起始处:
  \begin{itemize}
    \item \texttt{0x00000000} 处存放主堆栈指针 MSP 初值;
    \item \texttt{0x00000004} 处存放 Reset 异常入口地址,即 \texttt{Reset\_Handler} 的起始地址。
  \end{itemize}

  在本实验的仿真中,复位信号 \texttt{RSTn} 由 0 变为 1 后,Cortex-M0 首先通过 AHB-Lite 总线读取地址 \texttt{0x00000000} 和 \texttt{0x00000004},从 RAMCODE 中取得堆栈初值和 Reset 向量,然后将 PC(Program Counter,程序计数器)装载为 Reset 向量中给出的入口地址。此后的一段时间内,PC 会在 \texttt{Reset\_Handler} 所在的代码区间内顺序执行。

  因此,在实验波形中若观察到:
  \begin{itemize}
    \item 复位释放后,CPU 首先访问 \texttt{0x00000000} 和 \texttt{0x00000004} 等向量表地址;
    \item 随后 \texttt{vis\_pc\_o} 的值被设置为向量表中给出的 Reset 入口地址,并在这一区间内顺序执行;
  \end{itemize}
  即可判断此时 CPU 正在执行 Reset 异常服务程序,也就是处于 Reset 异常状态。

  \vspace{1em}

  \item \textbf{如何判断总线操作开始的表示是什么(哪几个信号的值决定了总线操作状态,分别是什么)?}

  本实验使用的是 AHB-Lite 片上总线协议(AHB-Lite bus protocol)。在该协议中,一次总线传输(Bus transfer)分为地址阶段(Address phase)和数据阶段(Data phase),并且采用流水线方式工作。当出现\textbf{一次新的、有效的总线传输开始}时,需要同时满足以下条件:

  \paragraph{(1) 由 HTRANS 表示传输类型}

  总线传输类型由 \texttt{HTRANS[1:0]} 信号决定:
  \begin{itemize}
    \item \texttt{HTRANS = 2'b00}:IDLE,无有效传输;
    \item \texttt{HTRANS = 2'b01}:BUSY,占用总线但不发起新传输;
    \item \texttt{HTRANS = 2'b10}:NONSEQ,新的非顺序传输,本实验中使用的主要类型;
    \item \texttt{HTRANS = 2'b11}:SEQ,顺序传输(Burst 传输中的后续传输)。
  \end{itemize}

  AHB-Lite 规定:\textbf{只有当 \texttt{HTRANS[1] = 1} 时,才表示本周期存在有效传输},即 \texttt{HTRANS} 为 NONSEQ 或 SEQ。本实验中仅使用 NONSEQ,因此可以认为:
  \[
      \texttt{HTRANS} = 2'b10 \quad \Rightarrow \quad \text{本周期为一次新的有效传输的地址阶段。}
  \]

  \paragraph{(2) 由 HREADY 表示上一笔传输是否完成}

  \texttt{HREADY} 为总线就绪信号(Ready signal),由所有从机的 \texttt{HREADYOUT} 综合得到,用于指示当前数据阶段是否完成:
  \begin{itemize}
    \item \texttt{HREADY = 1}:当前传输在本周期结束,可进入下一次传输;
    \item \texttt{HREADY = 0}:当前传输尚未完成,需要插入等待周期,地址和控制信号保持不变。
  \end{itemize}

  因此,只有当 \texttt{HREADY = 1} 时,主机在本周期输出的地址和控制信息才能被视作“新传输”的地址阶段。

  \paragraph{(3) 对某个具体从机,还需片选信号 HSELx 有效}

  总线扩展模块中的地址译码器(Address Decoder)根据 \texttt{HADDR} 产生各从机的片选信号 \texttt{HSELx},表示当前地址是否落在该从机的地址空间。例如 RAMCODE 的地址范围为 \texttt{0x00000000--0x0000FFFF},译码条件为:
  \[
      \texttt{HADDR[31:16]} = 16'h0000 \quad \Rightarrow \quad \texttt{HSEL\_RAMCODE} = 1.
  \]

  因此,一个具体从机的\textbf{一次有效总线访问}需满足:
  \[
    \text{有效访问某从机} \iff (\texttt{HSELx} = 1) \& (\texttt{HTRANS[1]} = 1) \& (\texttt{HREADY} = 1).
  \]

  \vspace{1em}

  \item \textbf{总线读操作时外设如果需要多个周期(例如 3 个周期)完成数据准备,那么 \texttt{Slave\_Interface} 中的哪个信号需要重新描述以及如何实现(简述时序)?}

  如果外设需要多个周期才能完成数据准备,那么在 AHB\mbox{-}Lite 总线的 Slave Interface 中,$\texttt{HREADYOUT}$(或由其组合得到的总线 $\texttt{HREADY}$)信号需要被重新描述,以实现等待状态(Wait State)的插入。

实现方式(时序简述)如下:

\begin{itemize}
  \item 初始状态:当主机(Master)发起读操作,地址和控制信号在总线上有效时,从机(Slave)通过检测
  $\texttt{HSEL}$、$\texttt{HTRANS[1]} = 1$ 以及 $\texttt{HREADY} = 1$ 等条件,识别出这是针对自己的有效读请求。

  \item 拉低 $\texttt{HREADYOUT}$:在数据尚未准备好的阶段,从机需要将 $\texttt{HREADYOUT}$ 信号驱动为低电平(0)。
  这会告诉主机,从机还没有准备好数据,需要主机延长当前数据阶段、暂时不要结束本次传输。

  \item 维持等待:如果外设需要 3 个周期来准备数据,那么从机需要在接下来的 2 个时钟周期内保持
  $\texttt{HREADYOUT}$ 为低电平。在这段等待期间:
  \begin{itemize}
    \item 主机必须保持地址和控制信号不变(即保持原来的 $\texttt{HADDR}$、$\texttt{HTRANS}$、$\texttt{HWRITE}$ 等值);
    \item 主机在每个时钟上升沿检查 $\texttt{HREADY}$ 信号($\texttt{HREADY}$ 信号通常是所有从机 $\texttt{HREADYOUT}$ 信号组合后的结果);
    \item 由于 $\texttt{HREADY} = 0$,本次数据阶段尚未完成,$\texttt{HRDATA}$ 上的数值不会被主机采样,可视为无效或保持上一次的值。
  \end{itemize}

  \item 准备就绪:在第 3 个时钟周期,当从机终于准备好数据并将\textbf{最终有效}的数据放置在
  $\texttt{HRDATA}$ 总线上时,需要在该周期的时钟上升沿到来之前,将 $\texttt{HREADYOUT}$ 信号驱动为高电平。
  此时要求 $\texttt{HRDATA}$ 在本周期内保持稳定,以满足主机对数据的建立/保持时间要求。

  \item 完成传输:主机在检测到 $\texttt{HREADY}$ 为高电平的那个时钟上升沿,会采样 $\texttt{HRDATA}$ 总线上的数据,
  从而在该时钟沿完成这次读操作的数据阶段。随后,主机可以在下一周期发起新的总线传输。
\end{itemize}

简单来说,$\texttt{HREADYOUT}$ 信号就是从机用来“暂停”总线传输的机制:当它为低时,当前数据阶段被延长,主机不会采样
$\texttt{HRDATA}$;当它重新变为高电平时,表示从机已经准备好数据,当前数据阶段会在这一时钟上升沿完成,主机在此时刻读取
$\texttt{HRDATA}$ 的有效数据。


\end{enumerate}
